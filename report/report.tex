\documentclass[acmtog]{acmart}
\usepackage{graphicx}
\usepackage{subfigure}
\usepackage{natbib}
\usepackage{listings}
\usepackage{bm}
\usepackage{amsmath}

\definecolor{blve}{rgb}{0.3372549 , 0.61176471, 0.83921569}
\definecolor{gr33n}{rgb}{0.29019608, 0.7372549, 0.64705882}
\makeatletter
\lst@InstallKeywords k{class}{classstyle}\slshape{classstyle}{}ld
\makeatother
\lstset{language=C++,
	basicstyle=\ttfamily,
	keywordstyle=\color{blve}\ttfamily,
	stringstyle=\color{red}\ttfamily,
	commentstyle=\color{magenta}\ttfamily,
	morecomment=[l][\color{magenta}]{\#},
	classstyle = \bfseries\color{gr33n}, 
	tabsize=2
}
\lstset{basicstyle=\ttfamily}

% Title portion
\title{Assignment 1:\\ {Exploring OpenGL Programming}} 

\author{Name: Chengyang Li\quad \\ student number: 2022533154\
\\email: \quad \texttt{lichy22@shanghaitech.edu.cn}}

% Document starts
\begin{document}
\maketitle

\vspace*{2 ex}

\section{Introduction}
[must] Compile the source code and configure the language server environment.

[must] Implement ray-triangle intersection functionality.

[must] Implement ray-AABB intersection functionality.

[must] Implement the BVH (Bounding Volume Hierarchy) construction.

[must] Implement the IntersectionTestIntegrator and PerfectRefraction material for basic ray tracing validation, handing refractive and solid surface interactions

[must] Implement a direct lighting function with diffuse BRDF and shadow testing.

[must] Implement anti-aliasing via multi-ray sampling per pixel within a sub-pixel aperture.

\section{Implementation Details}

This section provides detailed descriptions of the implementation locations and core algorithms for each task.

\subsection{Ray-Triangle Intersection (10\%)}

\textbf{Implementation Location:} \texttt{src/accel.cpp}, lines 74-123, function \texttt{TriangleIntersect}

\textbf{Algorithm Description:} Implements ray-triangle intersection using the Möller-Trumbore algorithm. This algorithm directly computes barycentric coordinates and intersection distance by solving a linear system.

\textbf{Key Steps:}
\begin{itemize}
    \item Compute triangle edge vectors $e_1 = v_1 - v_0$ and $e_2 = v_2 - v_0$
    \item Compute auxiliary vector $p = \text{dir} \times e_2$ and determinant $\text{det} = e_1 \cdot p$
    \item Compute barycentric coordinates $u = (t_{vec} \cdot p) / \text{det}$ and $v = (\text{dir} \cdot q) / \text{det}$
    \item Compute intersection distance $t = (e_2 \cdot q) / \text{det}$
    \item Verify boundary conditions: $u \geq 0, v \geq 0, u + v \leq 1, t \in [t_{min}, t_{max}]$
\end{itemize}

\subsection{Ray-AABB Intersection (10\%)}

\textbf{Implementation Location:} \texttt{src/accel.cpp}, lines 32-75, function \texttt{AABB::intersect}

\textbf{Algorithm Description:} Implements ray-AABB intersection using the Slab method. Treats the AABB as the intersection of three pairs of parallel planes and computes the ray intersection with each pair.

\textbf{Key Steps:}
\begin{itemize}
    \item Compute entry time $t_0$ and exit time $t_1$ for each of the X, Y, Z axes
    \item Handle cases where ray direction is negative (swap $t_0$ and $t_1$)
    \item Compute overall entry time $t_{in} = \max(t_{0x}, t_{0y}, t_{0z})$
    \item Compute overall exit time $t_{out} = \min(t_{1x}, t_{1y}, t_{1z})$
    \item Check intersection condition: $t_{out} \geq t_{in}$ and $t_{out} \geq 0$
\end{itemize}

\subsection{BVH Construction (25\%)}

\textbf{Implementation Location:} \texttt{include/rdr/bvh\_tree.h}, lines 127-214, function \texttt{BVHTree::build}

\textbf{Algorithm Description:} Implements recursive BVH tree construction using median-based heuristics for space partitioning.

\textbf{Key Steps:}
\begin{itemize}
    \item \textbf{Termination Condition (lines 136-155):} Create a leaf node when the number of nodes $\leq 1$ or depth $\geq$ CUTOFF\_DEPTH
    \item \textbf{Median Split (lines 178-186):}
    \begin{itemize}
        \item Select the longest dimension of the bounding box as the split axis
        \item Use \texttt{std::nth\_element} to perform median split based on centroid positions
        \item Divide the node array into two parts: $[\text{span\_left}, \text{split})$ and $[\text{split}, \text{span\_right})$
    \end{itemize}
    \item Recursively build left and right subtrees
    \item Merge the bounding boxes of subtrees as the parent node's bounding box
\end{itemize}

\subsection{IntersectionTestIntegrator (25\%)}

This task contains three sub-components:

\subsubsection{Multi-ray Sampling and Anti-aliasing}

\textbf{Implementation Location:} \texttt{src/integrator.cpp}, lines 41-61, function \texttt{IntersectionTestIntegrator::render}

\textbf{Implementation Description:}
\begin{itemize}
    \item Generate \texttt{spp} (samples per pixel) rays for each pixel
    \item Use \texttt{Sampler::getPixelSample()} to obtain sampling positions with sub-pixel random offsets
    \item Generate differential rays and compute radiance
    \item Submit results to the film, which automatically performs Monte Carlo integration averaging
\end{itemize}

\subsubsection{Refraction Ray Tracing}

\textbf{Implementation Location:} \texttt{src/integrator.cpp}, lines 92-109, function \texttt{IntersectionTestIntegrator::Li}

\textbf{Implementation Description:}
\begin{itemize}
    \item Use a loop to trace rays until hitting a diffuse surface or exceeding maximum depth
    \item Detect surface type (ideal diffuse, perfect refraction, or other)
    \item For perfect refraction surfaces, call the BSDF sampling function to obtain the refraction direction
    \item Use \texttt{spawnRay} to generate new rays to avoid self-intersection
    \item Continue tracing until finding a diffuse surface
\end{itemize}

\subsubsection{Direct Lighting Computation}

\textbf{Implementation Location:} \texttt{src/integrator.cpp}, lines 131-183, function \texttt{IntersectionTestIntegrator::directLighting}

\textbf{Implementation Description:}
\begin{itemize}
    \item \textbf{Shadow Testing (lines 136-151):}
    \begin{itemize}
        \item Cast shadow rays from the intersection point towards the light source
        \item Check for occluders
        \item If an occluder is found before the light source, return black (no lighting)
    \end{itemize}
    \item \textbf{Lighting Computation (lines 159-180):}
    \begin{itemize}
        \item Use Lambert diffuse model: $L = \Phi \cdot \rho \cdot \cos\theta / (4\pi d^2)$
        \item Where $\Phi$ is light flux, $\rho$ is albedo, $\theta$ is incident angle, $d$ is distance
        \item Obtain albedo through BSDF's \texttt{evaluate} function
    \end{itemize}
\end{itemize}

\subsection{Perfect Refraction Material (20\%)}

\textbf{Implementation Location:} \texttt{src/bsdf.cpp}, lines 90-114, function \texttt{PerfectRefraction::sample}

\textbf{Algorithm Description:} Implements physical simulation of perfect refraction and total internal reflection.

\textbf{Key Steps:}
\begin{itemize}
    \item Determine whether the ray is entering or leaving the medium based on the incident direction
    \item Adjust the normal direction to ensure it points to the correct hemisphere
    \item Use Snell's law to compute the refraction direction: $\eta_1 \sin\theta_1 = \eta_2 \sin\theta_2$
    \item Call the \texttt{Refract} function to attempt refraction
    \item If total internal reflection occurs ($\sin\theta_2 > 1$), use the \texttt{Reflect} function to compute the reflection direction
    \item Set \texttt{interaction.wi} to the computed direction
\end{itemize}

\subsection{Anti-Aliasing (5\%)}

\textbf{Implementation Location:} Integrated in the \texttt{IntersectionTestIntegrator::render} function

\textbf{Implementation Description:} Anti-aliasing is achieved through multiple random samples within each pixel. The sampler automatically generates random offsets within the pixel range $[x, x+1) \times [y, y+1)$, and the film automatically averages the results of multiple samples, effectively reducing aliasing artifacts.

\section{Results}

This section presents the rendering results of the implemented ray tracer.


\begin{figure}[htbp]
    \centering
    \includegraphics[width=0.5\textwidth]{../assignment3_files/cbox_no_light_refract.png}
    \caption{Cornell box scene with a refractive glass sphere. The scene demonstrates ray tracing with refraction, direct lighting, and anti-aliasing.}
    \label{fig:cbox_refract}
\end{figure}

\end{document}

